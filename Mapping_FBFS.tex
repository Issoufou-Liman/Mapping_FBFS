\documentclass[12pt,oneside]{article}
\usepackage[]{mathpazo}
\usepackage{amssymb,amsmath}
\usepackage{ifxetex,ifluatex}
\usepackage{fixltx2e} % provides \textsubscript
\ifnum 0\ifxetex 1\fi\ifluatex 1\fi=0 % if pdftex
  \usepackage[T1]{fontenc}
  \usepackage[utf8]{inputenc}
\else % if luatex or xelatex
  \ifxetex
    \usepackage{mathspec}
  \else
    \usepackage{fontspec}
  \fi
  \defaultfontfeatures{Ligatures=TeX,Scale=MatchLowercase}
\fi
% use upquote if available, for straight quotes in verbatim environments
\IfFileExists{upquote.sty}{\usepackage{upquote}}{}
% use microtype if available
\IfFileExists{microtype.sty}{%
\usepackage{microtype}
\UseMicrotypeSet[protrusion]{basicmath} % disable protrusion for tt fonts
}{}
\usepackage[margin=1in]{geometry}
\usepackage{hyperref}
\PassOptionsToPackage{usenames,dvipsnames}{color} % color is loaded by hyperref
\hypersetup{unicode=true,
            pdftitle={Application of Bayesian networks to mapping Flood-based Farming Systems.},
            pdfauthor={Issoufou Liman1,2,; Cory Whitney1,3; James Kungu2; Eike Luedeling1,3,4},
            colorlinks=true,
            linkcolor=blue,
            citecolor=red,
            urlcolor=blue,
            breaklinks=true}
\urlstyle{same}  % don't use monospace font for urls
\usepackage{graphicx,grffile}
\makeatletter
\def\maxwidth{\ifdim\Gin@nat@width>\linewidth\linewidth\else\Gin@nat@width\fi}
\def\maxheight{\ifdim\Gin@nat@height>\textheight\textheight\else\Gin@nat@height\fi}
\makeatother
% Scale images if necessary, so that they will not overflow the page
% margins by default, and it is still possible to overwrite the defaults
% using explicit options in \includegraphics[width, height, ...]{}
\setkeys{Gin}{width=\maxwidth,height=\maxheight,keepaspectratio}
\IfFileExists{parskip.sty}{%
\usepackage{parskip}
}{% else
\setlength{\parindent}{0pt}
\setlength{\parskip}{6pt plus 2pt minus 1pt}
}
\setlength{\emergencystretch}{3em}  % prevent overfull lines
\providecommand{\tightlist}{%
  \setlength{\itemsep}{0pt}\setlength{\parskip}{0pt}}
\setcounter{secnumdepth}{5}

%%% Use protect on footnotes to avoid problems with footnotes in titles
\let\rmarkdownfootnote\footnote%
\def\footnote{\protect\rmarkdownfootnote}

%%% Change title format to be more compact
\usepackage{titling}

% Create subtitle command for use in maketitle
\providecommand{\subtitle}[1]{
  \posttitle{
    \begin{center}\large#1\end{center}
    }
}

\setlength{\droptitle}{-2em}

  \title{Application of Bayesian networks to mapping Flood-based Farming Systems.}
    \pretitle{\vspace{\droptitle}\centering\huge}
  \posttitle{\par}
    \author{Issoufou Liman\textsuperscript{1,2,*} \\ Cory Whitney\textsuperscript{1,3} \\ James Kungu\textsuperscript{2} \\ Eike Luedeling\textsuperscript{1,3,4}}
    \preauthor{\centering\large\emph}
  \postauthor{\par}
      \predate{\centering\large\emph}
  \postdate{\par}
    \date{Caamsa 20, 2019}

% \documentclass{article} https://tex.stackexchange.com/questions/60209/how-to-add-an-extra-level-of-sections-with-headings-below-subsubsection
\makeatletter
% Editing the default numbering style
\renewcommand{\thesection}{\Roman{section}.}
\renewcommand{\thesubsection}{\thesection\arabic{subsection}.}
\renewcommand{\thesubsubsection}{\thesubsection\arabic{subsubsection}.}

\renewcommand\paragraph{\@startsection{paragraph}{4}{\z@}%
            {-2.5ex\@plus -1ex \@minus -.25ex}%
            {1.25ex \@plus .25ex}%
            {\normalfont\normalsize\bfseries}}   

% \renewcommand\subparagraph{\@startsection{subparagraph}{5}{\z@}%
%             {-1.25ex\@plus -1ex \@minus -.30ex}%
%             {1.00ex \@plus .30ex}%
%             {\normalfont\normalsize\bfseries}}
            
\renewcommand\subparagraph{\@startsection{subparagraph}{5}{\z@}%
            {-1.25ex\@plus -1ex \@minus -.30ex}%
            {1.00ex \@plus .30ex}%
            {\normalfont\normalsize\bfseries\itshape}}  
            
\renewcommand{\theparagraph}{\thesubsubsection\arabic{paragraph}.}
\renewcommand{\thesubparagraph}{\theparagraph\arabic{subparagraph}.}


\makeatother
\setcounter{secnumdepth}{3} % how many sectioning levels to assign numbers to
\setcounter{tocdepth}{5}    % how many sectioning levels to show in ToC

\begin{document}
\maketitle
\begin{abstract}
Accurate information on actual areas under given cropping/farming
systems is an important input for applied agricultural research and
policies. Therein, remote sensing provides a useful tool for analysing
spatial metrics in an explicit manner. In this paper, we captured
various characteristics of flood-based farming systems (FBFS) with
remote sensing and applied a probabilistic framework for system
analysis. FBFS were mapped across the Kisumu County in Kenya by
incorporating uncertainties into mapping overlays. A Bayesian network
(BNs) was used to describe and interlink various features of FBFS using
expert knowledge. 3 years (2014-2016) of high-level MODIS VI products
and LANDSAT OLI-TIRS raster data were acquired in the forms of
normalized difference spectral indices (NDSI). These were used as
proxies for estimating various metrics of FBFS which were later used to
feed the spatial data nodes of the BNs and were propagated along with
the rest of other non-spatial nodes derived from expert judgement. We
demonstrate that spatial explicit information can be derived from remote
sensing data as fuzzy linguistic quantifiers which are suitable for
representing node states in BNs. When such metrics are available, BNs
are useful tools for incorporating uncertainties when mapping complex
systems in a context of limited and uncertain information. The causal
probabilistic reasoning embedded in the approach was tested on the
Tigray region in Ethiopia; an area with totally different settings; and
seems to perform incredibly well. Results were transparently generated
in forms of intermediary prior spatial maps for specific metrics to
ultimately be aggregated into final posterior maps of plausible areas
for flood-based farming along with their spatial explicit uncertainties.
\end{abstract}

{
\hypersetup{linkcolor=black}
\setcounter{tocdepth}{5}
\tableofcontents
}
\textsuperscript{1} World Agroforestry Centre (ICRAF), United Nations
Avenue, Gigiri, P. O. Box 30677-00100, Nairobi, Kenya\\
\textsuperscript{2} Kenyatta University, Department of Environmental
Sciences\\
\textsuperscript{3} Center for Development research (ZEF), University of
Bonn, Genscherallee 3, D-53113, Bonn, Germany\\
\textsuperscript{4} University of Bonn, Department of Horticultural
Sciences, Auf dem Hügel 6, D-53121, Bonn, Germany

\textsuperscript{*} Correspondence:
\href{mailto:issoufoul@gmail.com}{Issoufou Liman
\textless{}\href{mailto:issoufoul@gmail.com}{\nolinkurl{issoufoul@gmail.com}}\textgreater{}}

\hypertarget{introduction}{%
\section{Introduction}\label{introduction}}

Flood-based farming systems (FBFS) are rainfed farming systems occurring
in drylands areas and receiving extra irrigation from various type of
non-harmful floods that share the properties of being unpredictable, of
short duration, of low annual frequency and occurring in relatively
lowlands areas with gentile local topography where various social and
institutional arrangements govern the water access and
sharing.\textsuperscript{\protect\hyperlink{ref-Puertas_et_al_2011}{1}--\protect\hyperlink{ref-Varisco_1983}{3}}
By making flood water available for use in agriculture, these farming
systems contribute to food security along with many other benefits for
drylands'
society.\textsuperscript{\protect\hyperlink{ref-VanSteenbergen_et_al_2010}{2},\protect\hyperlink{ref-Xing_et_al_2014}{4}}
They are mostly extensive subsistence-based but support millions of
farmers especially in East Africa where they also provide income and
various other intangible
services.\textsuperscript{\protect\hyperlink{ref-Puertas_et_al_2011}{1}}
Many countries across Africa and Asia endorsed their development through
a common framework for research, policy and action (i.e.~Flood-based
Livelihood Network (FBLN) foundation) mandated for knowledge sharing,
filling critical knowledge gaps and improving their productivity of
these farming
systems.\textsuperscript{\protect\hyperlink{ref-Puertas_et_al_2011}{1},\protect\hyperlink{ref-FBLN_2018}{5}}
Despite their promises and importance, FBFS have been the topic of
surprisingly few
studies\textsuperscript{\protect\hyperlink{ref-VanSteenbergen_et_al_2010}{2}}
focusing on aspects related to hydrology and sedimentation, design and
maintenance, or sociology attuned to specific
contexts.\textsuperscript{\protect\hyperlink{ref-VanSteenbergen_et_al_2010}{2},\protect\hyperlink{ref-Haile_2010}{6}}
Consequently, there is a general lack of information to answer relevant
questions for FBLN countries. For example, little effort has been made
to provide reliable estimates of their
coverage.\textsuperscript{\protect\hyperlink{ref-VanSteenbergen_et_al_2010}{2}}
Despite the existing country-specific area estimates, it is often
unclear how these were derived resulting in large uncertainties across
these
estimates.\textsuperscript{\protect\hyperlink{ref-VanSteenbergen_et_al_2010}{2}}
Most FBLN countries' reports expressed the need for mapping FBFS, yet
only few
attempts\textsuperscript{\protect\hyperlink{ref-Ghebreamlak_et_al_2018}{7}--\protect\hyperlink{ref-Theilen-Willige_et_al_2015}{9}}
are available towards this end in our knowledge. While most aspects
related to FBFS are
unpredictable,\textsuperscript{\protect\hyperlink{ref-Puertas_et_al_2011}{1},\protect\hyperlink{ref-VanSteenbergen_et_al_2010}{2}}
these approaches seems to rely on the assumption that agronomic flooding
can be clearly detected on a satellite image in the near time of
flooding event and sometimes imply that the areas to be mapped and the
timing of the flooding is already known and matches a remote sensing
scene in space and time. Furthermore, some of these approaches rely on
the derivation of some kind of thresholds based on which FBFS can be
discriminated. Threshold-based methods may be impractical considering
the complexity of FBFS, the communalities they share with both forests,
open water, rainfed and irrigated
systems,\textsuperscript{\protect\hyperlink{ref-Puertas_et_al_2011}{1},\protect\hyperlink{ref-VanSteenbergen_et_al_2010}{2}}
and particularly when discriminating between different type of
FBFS.\textsuperscript{\protect\hyperlink{ref-Boschetti_et_al_2014}{10}}
Moreover, some of these methods rely on visual interpretations which are
hard to upscale and may require a keen knowledge of the area. While it
is challenging to map these FBFS, accurate information is needed to
monitor them, assess their contribution to food security, biodiversity
conservation and other intangible benefits at various scales, urging the
need for standard methodologies for mapping them. We argue such a
surveillance can be best achieved using Geoinformation.
\protect\hyperlink{ref-Wegmann_et_al_2016}{11} covered a range of
capabilities and applications of remote sensing in development research,
including the use of spectral vegetation indices, and digital elevation
models in ecological studies which have potentials in mapping FBFS. Even
though satellite remote sensing provides a framework for comprehensive
mapping of these farming
systems,\textsuperscript{\protect\hyperlink{ref-Wegmann_et_al_2016}{11},\protect\hyperlink{ref-Konecny_2014}{12}}
traditional methods may fail to discriminate them due to their
complexity and diversity across different
settings.\textsuperscript{\protect\hyperlink{ref-Boschetti_et_al_2014}{10}}
The complexity arises from their similarities with both rainfed and
irrigated
systems\textsuperscript{\protect\hyperlink{ref-Puertas_et_al_2011}{1},\protect\hyperlink{ref-VanSteenbergen_et_al_2010}{2}}
along with a number of their biophysical attributes that could exhibit
different signatures and can, therefore, only provide partial
information for discriminating them. In flood-based agriculture, one may
expect, for instance, an open water followed by rapid increase in
vegetation or a combination of
thereof.\textsuperscript{\protect\hyperlink{ref-Boschetti_et_al_2014}{10}}
The sequence open water-vegetation, though, can be misleading because
the flood event can take place without leaving detectable trace of
inundation not only because the sensor can miss to capture the flood
event but also because many FBFS have deep soils capable of storing
large amount of water. The flood event may also occur after the planting
date in which case it can be confused with other flood hazards. The
diversity of flood-based agriculture may pose another challenge for
mapping FBFS because locally adopted thresholds may be
required\textsuperscript{\protect\hyperlink{ref-Bashari_et_al_2008}{13}}
due to the diversity of conditions under which they are
practiced.\textsuperscript{\protect\hyperlink{ref-Puertas_et_al_2011}{1},\protect\hyperlink{ref-VanSteenbergen_et_al_2010}{2}}
In addition, the areas under the practice may be unstable over time
because of changes in the water courses, water flow,
etc..\textsuperscript{\protect\hyperlink{ref-VanSteenbergen_et_al_2010}{2}}
There is clearly a need for generic approaches for mapping FBFS which
can be applied to several contexts given their diversity and the
diversity of crops grown under
them.\textsuperscript{\protect\hyperlink{ref-Puertas_et_al_2011}{1}}
Mapping such complex systems, where events can be causally related in a
context of data uncertainty may require multivariate approaches that are
able to leverage on both available information and
expertise.\textsuperscript{\protect\hyperlink{ref-Hubbard_2014}{14}--\protect\hyperlink{ref-Whitney_et_al_2018}{17}}
Considering the data uncertainty for remote sensing of FBFS, Such an
approach, conversely, needs to be able to combine knowledge from domain
experts with learning from all available data, be suitable for
incomplete information while accounting for the sources of
uncertainties.\textsuperscript{\protect\hyperlink{ref-Kuhnert_2011}{18}--\protect\hyperlink{ref-Whitney_et_al_2018a}{20}}
These expectations can be achieved using
BNs.\textsuperscript{\protect\hyperlink{ref-Whitney_et_al_2018a}{20},\protect\hyperlink{ref-Yet_et_al_2016}{21}}
The technicalities along with the many real-world applications of BNs
are beyond the scope of this paper and are extensively documented
elsewhere.\textsuperscript{\protect\hyperlink{ref-Pourret_et_al_2008}{22}}
The objective of this paper is to provide an affordable mapping routine
that could be used to comprehensively estimate the coverage of FBFS.
However, the concept is reproducible in many other contexts where
spatial data are available. Based on the definition and characteristics
of
FBFS,\textsuperscript{\protect\hyperlink{ref-Puertas_et_al_2011}{1},\protect\hyperlink{ref-VanSteenbergen_et_al_2010}{2},\protect\hyperlink{ref-VanSteenbergen_et_al_2011}{23}}
examining a number of biophysical surface states related to their
characteristic hydrology, vegetation, and topography, preferably in
forms of time series where possible, are clearly relevant for mapping
them. With this assumption, we used open source satellite remote sensing
data and software to capture various FBFS attributes following a
conceptual model developed by experts using various normalized
difference spectral indices (NDSI) and topographic
structures\textsuperscript{\protect\hyperlink{ref-Wegmann_et_al_2016}{11},\protect\hyperlink{ref-Arge_et_al_2003}{24}--\protect\hyperlink{ref-Yang_et_al_2006}{34}}
specific to FBFS to ultimately derive their probability of occurrence
for each pixel across Kisumu County in Kenya. We argue that
threshold-based methods may be unsuitable to mapping such complex
systems and demonstrate the versatility of probabilistic methods.
Despite the ranges of data uncertainty under such complex systems, we
elaborate on the existence of some sort of spatio-temporal pattern in
the general landscape that can described using causality and from which
fuzzy metrics can be derived to describe distinctives sub-entities. We
described how this can be achieved and by deriving the required inputs
for BNs from raw spatial data. The approach was tested across the Tigray
region in Ethiopia (not presented here); an area with different
settings; and seems to perform incredibly well. The core data used in
the study include three years (2014-2016) of high-level Moderate
Resolution Imaging Spectroradiometer (MODIS) version VI products
acquired in the forms of NDSI, the Shuttle Radar Topographic Mission
(STRM) void-filled Digital Elevation Model (DEM). Data analysis were
conducted using the R programming
language\textsuperscript{\protect\hyperlink{ref-RCoreTeam_2018}{35}} as
much as possible and software packages are referred to as such unless
otherwise specified.

The study took place in two regions of Kenya and Ethiopia, the Kisumu
county and the Tigray region respectively. Both areas have long history
in terms of the practice of flood-based agriculture and have different
ways of managing agronomic flooding. However, this paper specifically
focuses on the case of Kisumu to provide the backbones of the
methodology. Kisumu is one of the largest of Kenya's counties, located
in the western
Kenya.\textsuperscript{\protect\hyperlink{ref-KisumuCountyGovernment_2013}{36}}
It is bordered by the Homa Bay County to the South, Nandi County to the
North East, Kericho County to the East, Vihiga County to the North West
and Siaya County to the West. Kisumu County extends between 33° 20'E and
35° 20'E and 0° 20'S and 0° 50'S covering an area of 2576.5 km2 of which
567 km2 (4.5\%) is covered by
water.\textsuperscript{\protect\hyperlink{ref-KisumuCountyGovernment_2013}{36}}
Common topographic features are the Winam Gulf of Lake Victoria and the
Rift valley. The central part of the county is relatively flat lands
surrounded by higher altitudinal ridges reaching up to 1835 m. The
relief of the county can be described by 3 main topographical features,
the Kano lowlands plains, the Maseno midlands, and the highlands of
Nyabondo Plateau making most of the central part prone to flooding,
particularly during periods of heavy
rains.\textsuperscript{\protect\hyperlink{ref-KisumuCountyGovernment_2013}{36}}
Kisumu county is endowed in surface water with three main rivers flowing
into Lake
Victoria.\textsuperscript{\protect\hyperlink{ref-KisumuCountyGovernment_2013}{36}}
The floodplains are good for agriculture due to their relatively rich
soils resulting from recurrent alluvial deposits and potential for
irrigation. Sensors' Characteristics SRTM SRTM is an collaborative
project between the US National Aeronautics and Space Administration
(NASA) and the National Imagery and Mapping Agency (NIMA), the National
Geospatial-Intelligence Agency, and the German and Italian Space
Agencies providing the most complete highest resolution and public
domain DEM between 56° S and 60°
N.\textsuperscript{\protect\hyperlink{ref-Farr_et_al_2000}{37}--\protect\hyperlink{ref-Nikolakopoulos_et_al_2006}{39}}
It uses a dual radar antennas interferometry to produce high resolution
topographic data and carries over 300 high-density tapes storing these
data. several spatial resolutions (e.g.~30m, 90m, 1000m) are
available.\textsuperscript{\protect\hyperlink{ref-Farr_et_al_2007}{38},\protect\hyperlink{ref-Jarvis_et_al_2008}{40}}
A list of available SRTM data and their sources along with various other
information can be found in Farr et
al.~\textsuperscript{\protect\hyperlink{ref-Farr_et_al_2007}{38}} and
the accuracy of the SRTM data is extensively discussed in Falorni et
al.~\textsuperscript{\protect\hyperlink{ref-Falorni_et_al_2005}{41}}. In
this study, we used the 30 m spatial resolution DEM on which further
processing were applied to ultimately compute the spatial data nodes
used in the BNs. MODIS MODIS is a key remote sensing sensor on both
Earth Observing System (EOS)-AM1 and EOS-PM1 operating onboard Terra and
Aqua satellites
respectively.\textsuperscript{\protect\hyperlink{ref-King_et_al_1995}{42}}
MODIS acquires data in the visible and infrared regions of the
electromagnetic spectrum to provide 36 spectral bands with a viewing
swath width of 2,330 km for studying various aspects of terrestrial
ecosystems. MODIS scenes are geo-localised using a sinusoidal projection
system and therefore identified using their specific horizontal (h) and
vertical (v) tiles number. These data are provided in different spatial
(250 m, 500 m, and 1 Km) and temporal (daily, 8 days, 16 days, etc.)
resolutions. In this study, we used the MODIS product MOD09A1 (Terra
Surface Reflectance 8-Day L3 Global 500m SIN Grid
V006\textsuperscript{\protect\hyperlink{ref-Vermote_2015}{43}} which
provides an 8-day composite atmospherically corrected land surface
reflectance across bands 1 to 7 along with other additional quality
indicators layers assessing the quality of the acquisition at pixel
level. Data acquisition and pre-processing The BNs BNs are reasoning
engine and knowledge engineering tools that can be used to model
partially known events in forms of probability, incorporate causal
uncertainty in the
analysis,\textsuperscript{\protect\hyperlink{ref-Pourret_et_al_2008}{22}}
and therefore allow reasoning under uncertainty. On the one hand, they
are causal probabilistic models, on the other hand, they are ways of
decomposing large joint probability
distributions\textsuperscript{\protect\hyperlink{ref-Pourret_et_al_2008}{22},\protect\hyperlink{ref-Pearl_2000}{44}}
and specifying complex heuristics that go beyond a simple regression
model.\textsuperscript{\protect\hyperlink{ref-Whitney_et_al_2018}{17}}
In Bayesian analysis, scientific reasoning starts with the current state
of knowledge and subsequently assesses the contribution of further new
information to that
knowledge.\textsuperscript{\protect\hyperlink{ref-Pearl_2000}{44}} BNs
become more exciting and have proven to give incredibly remarkable
results when there is a need to incorporate many variables in a model
and determine their effects on an outcome. In this study, the ultimate
outcome variable (PixelIsFBFS) determining whether a given pixel qualify
for FBFS is described using causal relationships among several other
variables specified in a BNs model (Figure 1). The conceptual model
described pixels' potential for FBFS (PixelIsFBFS) to be jointly defined
by the amount of flood water received (FloodSomePts), the topographic
suitability, and the extent of vegetation variation due to flood
(VegVarFlood) at that pixel. A pixel may record important run-on without
experiencing flood because the water may run-off without a suitable
topography (suiteTopo). This is, vegetation variation in space and time
can be due to either flood or non-flood water but each pixel can be
profiled in terms of its amount of flood water.

Figure 1. Bayesian Network describing the causal reasoning used for
mapping FBFS in Kisumu, Kenya. The topographic suitability of a pixel
(suiteTopo) depends largely on its slope (Slope) and the number of
pixels draining into it (FlowAcc). While the accumulation of flow
(FlowAcc) is important to the amount of flood reaching the pixels, it is
rather the slope (Slope) and the gravity that determine whether the
water will be kept or transmitted by that pixel. The extent to which
vegetation variation is caused by flood water (VegVarFlood) can be a
straightforward indicator for mapping flood-based agriculture but also
difficult to measure. Vegetation variation caused by flood (VegVarFlood)
was therefore assessed via proxy variables. For the temporal variation
in vegetation indices (TempoVarInVi) to be attributed to flood to some
extent (VegVarFlood) at given pixel, one may expect a certain degree of
flow accumulation (FlowAcc), and a certain likelihood of flood events
near the beginning of the growing season (FloodBgs). Then, the temporal
variation in vegetation (TempoVarInVi) of the pixel is assumed to be
influenced by a number of factors such as its soil water content
(SoilwaterCont), its exposure (Expos2wet) and sensitivity to flood
(Sens2Flood), or even the quality of the data in relation with the type
of algorithm used in processing these data (PowerTools). For a given
pixel to be flooded at some points (FloodSomePts), there must be water
presence either on the ground (WaterPrznt). This is moderated by the
soils suitability (SuitSoil) resulting in specific type of vegetation
(CharactVeg). Therefore, the extent to which a pixel is flooded
(FloodSomePts) can be deduced from its typical vegetation (CharactVeg),
the soil suitability (SuitSoil), and the presence of water (WaterPrznt).
Soil suitability to FBFS is defined by its exposure to wetness
(Expos2wet) and more importantly its water content (SoilWaterCont).
Water presence (WaterPrznt) may depend on soil water content
(soilWaterCont), the power of tools (PowerTools), and the sensitivity to
flood (Sens2Flood). Characteristic FBFS' vegetations are assumed to be
due to vegetation sensitivity to water variation (Veg2waterVar) and the
sensitivity of the area to flood (Sens2Flood).

The shapefiles data The shapefiles of the study areas were acquired from
the global administrative boundaries
database\textsuperscript{\protect\hyperlink{ref-GADM_2018}{45}} as level
1 product, using the string country name as argument to \texttt{getData}
function from the raster
package,\textsuperscript{\protect\hyperlink{ref-Hijmans_2019}{46}} from
which we extracted the shapefile of the area of interest. These
shapefiles were latter used to identify the MODIS H/V titles
corresponding to the area. Other shapefiles used are the spatial
polygons representing well known selected surface features (land use/
land cover) used to understand the behaviour of the spectral response in
the area. These were either collected during fields works using handheld
GPS devices or digitized on Google Earth to be used to query the
spectral response of pixels they match spatially. The surfaces features
considered here include different depth water bodies, settlements, FBFS
fields, rainfed agricultural fields, riparian forests, forests with
varying degree of density, and sugarcane fields having relatively
extended growing season. MODIS Tera We considered a period of three
years (2014 - 2016) and queried all available MODIS Tera data
overlapping the acquired shapefiles of the area of interest (section
I.3.2). We first determined the Horizontal and Vertical tiles
corresponding to the spatial coverage of the shapefiles and used these
to define the appropriate geo-locations of the study area relative to
the MODIS Sinusoidal grid systems using the \texttt{getTiles} function
from the MODIS
package.\textsuperscript{\protect\hyperlink{ref-Mattiuzzi_and_Detsch_2018}{47}}
As result, the Kisumu County falls in-between the h21v09 and h21v08.
Based on these tiles and the temporal window of interest, we mosaiced
the images and calculated the NDSIs after filtering and downloading the
data from MODIS global online database (i.e.~LP DAAC, LAADS) using the
machine-to-machine routines embedded in the \texttt{MODIStsp} function
from the package of the same
name.\textsuperscript{\protect\hyperlink{ref-Busetto_and_ranghetti_2016}{48}}
Since the MODIS data are provided in hierarchical data format (hdf),
they were re-projected to World Geodetic System 1984 (WGS 84) and
subsequently translated into Geotiff format using GDAL
2.2.3.\textsuperscript{\protect\hyperlink{ref-GDAL_OGRcontributors_2018}{49}}
Making sense of data in the context Phenological overview of the
landscape texture We sampled approximatively 40\% of pixels considering
a regular grid sampling using the \texttt{spSample} function from the sp
package.\textsuperscript{\protect\hyperlink{ref-Bivand_et_al_2013}{50},\protect\hyperlink{ref-Pebesma_Bivand_2005}{51}}
These time series of NDSIs were examined using boxplots. The NDSIs
considered include the Normalized Difference Vegetation
index,\textsuperscript{\protect\hyperlink{ref-Rouse_et_al_1973}{30},\protect\hyperlink{ref-Tucker_1979}{33}}
the Normalized Difference Flood
index,\textsuperscript{\protect\hyperlink{ref-Boschetti_et_al_2014}{10}}
the Goa's Normalized Difference Water
index,\textsuperscript{\protect\hyperlink{ref-Gao_1996}{25},\protect\hyperlink{ref-Ji_et_al_2009}{28}}
and the mcfeeters'
NDWI.\textsuperscript{\protect\hyperlink{ref-Ji_et_al_2009}{28},\protect\hyperlink{ref-McFeeters_1996}{29}}
These NDSIs were chosen due to their suitability in describing water and
vegetation.\textsuperscript{\protect\hyperlink{ref-Boschetti_et_al_2014}{10},\protect\hyperlink{ref-Gao_1996}{25},\protect\hyperlink{ref-Ji_et_al_2009}{28},\protect\hyperlink{ref-McFeeters_1996}{29},\protect\hyperlink{ref-Tucker_1979}{33}}
Spatio-temporal patterns The boxplots were supported with further
principal components
analysis\textsuperscript{\protect\hyperlink{ref-Mardia_et_al_1979}{52},\protect\hyperlink{ref-Pearson_1901}{53}}
using a customized version of the \texttt{rasterPCA} function from the
RStoolbox
package\textsuperscript{\protect\hyperlink{ref-Leutner_et_al_2019}{54}}
making used of the \texttt{prcomp} function from the base
package\textsuperscript{\protect\hyperlink{ref-RCoreTeam_2018}{35}} and
other low-level functions from the raster
package.\textsuperscript{\protect\hyperlink{ref-Hijmans_2019}{46}} The
PCA, used to interpret the large dataset by creating a smaller number of
components, was followed by other probabilistic estimations (see section
I.5) defining the spatial representations of the spatial data node
states to be used in the BNs. These supporting analyses focus only on
the NDVI and NDFI but a comparative analysis connecting all the NDSIs
across the main land uses is provided. The outputs were presented on the
map to give insights on how the states of vegetation and water are
spatially distributed. PCA allowed us to meaningfully interpret the
large dataset by creating a smaller number of components. Derivation of
spatial data nodes General Procedure Time series data The boxplots and
the comparative analysis of the NDSIs shows different patterns but these
can only be hardly separated making it difficult to threshold the
different NDSIs across land use / land cover. An approach based on rigid
thresholding was clearly not the best option to demarcate the nature of
the spectral response, at least not without making complicated
statistical assumptions. A simpler way would be to approach the problem
using imprecise quantifiers where fuzzy logic can be used in forms of
Likert-type
scales\textsuperscript{\protect\hyperlink{ref-Likert_1932}{55}}
describing the scores of the spectral responses along a range;
supporting the use of BNs. We therefore make use of the different ranges
of the boxplot (i.e.~lowest outliers -- lower Whisker, lower Whisker -
1st quartile, 1st quartile -- 3rd quartile, 3rd quartile - upper Whisker
, upper Whisker -- highest outliers) to separate the data and estimate
the probability of each pixel falling within each of these ranges at
each time step in the time series (Figure 5). Although rarely use for
deriving variable states, boxplot has been used in all kind of
statistics for the last 40 years. Beyond their incredible simplicity,
boxplots use robust statistics to summarise data in terms of
five-numbers while being particularly handy when comparing the
distributions of these data across
groups.\textsuperscript{\protect\hyperlink{ref-Wickham_and_Stryjewski_2012}{56}}To
estimate the probability for a given range, we first recoded all values
outside of that range to 0 whereas the values falling within the range
were recorded to 1. This way, all the real-valued time series was
reclassified into presence - absence time series data. Thereafter, the
final probability of a pixel belonging to a range can be easily
estimated using the ratio between the presence and absence of that pixel
in the range throughout the time series (eq. 1). P(p\_(i ϵ
〖range〗\emph{j ) )=n}(p\_i ϵ 〖range〗\emph{j )/N (eq.1) Where P(p}(i
ϵ 〖range〗\emph{j ) ) is the probability of the pixel i belonging to
the range j, n}(p\_i ϵ 〖range〗\_j ) is the number of time the pixel i
belonged to the range j, and N is the sample space (total size of the
time series). After estimating the probabilities for each boxplot range,
these ranges were mapped into fuzzy linguistic quantifiers in forms of
low, medium, high etc. (depending on the number of states identify by
the boxplot algorithm) to provide vegetation states in a spatially
explicit manner. For example, in situation where all the five boxplot
ranges are present, lower outlier values were mapped to very low, the
lower Whisker to low, the interquartile range to medium, the upper
Whisker to high and the upper outliers to very high. These states,
computed in separated raster layers, were then merged into one single
layer to provide discrete states on which the Bayesian network can be
easily operationalized. The decision to assign a state value to a given
pixel was made transparently using fully probabilistic heuristics in 3
sequential steps. The first step consisted in assigning a particular
state to a given pixels if that state scored the highest probability and
only if that highest probability is exclusive. Thus, all cases where 2
or more highest probability co-exist are considered as unknown and
ignore until the second step. Such cases having equal probability states
were commonly encountered at the edges where two consecutive states
spatially meet. In the second step, these confusing pixels having 2 or
more maximal probability of state occurrence are filled in using
majority rule. Herein, since spatial feature are geographically
auto-correlated in general, we applied a 3 by 3 moving window (eight-way
connectedness) centred on each of these uncertain state value pixels and
estimate their state value based on the values of the 8 neighbourhood
pixels. Similarly, to the first step, a particular value among the
values of the 8 neighbours is chosen when and only when it is the only
value that occurs the most. Whenever 2 or more states happen to be the
most probable at the same time, the pixel is left out and handled at the
third step. In the third step, the rest of these confusing pixels, where
the first and second steps heuristics failed, were modelled using
decision tree recursive partitioning classification algorithm. We first
extracted the maximal values which correspond to the highest probability
across all these remaining confusing pixels to fit a classification
regression tree (CART) model and used that model to conditionally infer
their state values. Therefore, after this third step all pixel state
value were estimated, and the resulting raster can be used as spatial
data node in the BNs model. The boxplot statistics were computed using
our own defined function based on \texttt{boxplot.stats} function from
the grDevices
package.\textsuperscript{\protect\hyperlink{ref-RCoreTeam_2018}{35}} We
used the \texttt{reclassify} and the \texttt{calc} functions both from
the raster
package\textsuperscript{\protect\hyperlink{ref-Hijmans_2019}{46}} to
recode the pixel values and to estimate the probability respectively.
The second step confusing pixels were handled using \texttt{modal}
function from within the \texttt{focal} function also from the raster
package.\textsuperscript{\protect\hyperlink{ref-Hijmans_2019}{46}} The
CART model was fitted using the \texttt{ctree} function from the party
package\textsuperscript{\protect\hyperlink{ref-Hothorn_et_al_2006}{57}}
whereas the third step confusing pixels' inference was achieved by
supplying an anonymous function wrapping the \texttt{predict} function
from the stats
package\textsuperscript{\protect\hyperlink{ref-RCoreTeam_2018}{35}} as
argument to \texttt{calc}
function.\textsuperscript{\protect\hyperlink{ref-Hijmans_2019}{46}}
Single layer data In the case of single layer data where time series
does not make sense (e.g.~slope and flow accumulation) or where nodes
were derived from times series as single layer (e.g.~exposure to
wetness, vegetation sensitivity to water), a slightly different
procedure compared to that adopted for time series data (section I.5)
was adopted for Bayesian network nodes making. We rather used the
boxplot's five-numbers approach in a much straightforward way by
exploiting the spatial variability across cell values since the temporal
component of this variability no longer exists. We first extracted all
pixel values and computed the five-numbers from which ranges
corresponding to discrete states were computed. These ranges were then
directly recoded into unique values mapping to fuzzy linguistic
quantifiers. The process is somewhat a simple raster reclassification
task except that the reclassification matrix was derived directly from
the natural breaks of the data. Specific procedure Soil water content
The node soil water content, as it says, estimates the probability of
water in soil and is therefore related to soil water holding capacity.
This was computed as a composite index by aggregating the
NDII6,\textsuperscript{\protect\hyperlink{ref-Hunt_and_Rock_1989}{26}}
the
NDII7,\textsuperscript{\protect\hyperlink{ref-Hunt_and_Rock_1989}{26}}
the NDVI,\textsuperscript{\protect\hyperlink{ref-Tucker_1979}{33}} and
the
NDFI.\textsuperscript{\protect\hyperlink{ref-Boschetti_et_al_2014}{10}}
While the NDII6 and the NDII7 perform well in detecting a mixture in
different compartments (e.g.~plant water, soil water, open water etc.)
due to the presence of NIR and SWIR bands, the NDFI gives an estimate of
flood conditions whereas the NDVI assesses the presence and condition of
vegetation. Combining these in one composite index may provide a
powerful tool for ecological studies since one index can be extracted
from another to provide a new
metric.\textsuperscript{\protect\hyperlink{ref-Boschetti_et_al_2014}{10}}
For example, knowing that the NDII6 broadly assess soil and vegetation
water content, one could extract the NDVI to estimate the soil water
content. Likewise, the NDFI can be extracted to provide an estimate of
vegetation water content. We used such assumption to estimate the soil
water content using the eq. 2. Such tricks where the original meaning of
an index is altered by subtracting another index is commonly used in
remote sensing and is by far the back bone of the spectral index theory.
〖SWC〗\_i=1/2*(∑\_1\^{}n▒〖NDII6〗\_i +∑\_1\^{}n▒〖NDII7〗\_i +2*
∑\_1\^{}n▒〖〖NDFI〗\_i- 2* ∑\_1\^{}n▒〖NDVI〗\_i )〗) (eq.2) Where SWCi
is the soil water content, and n the total length of the time series.
For each of these 4 indices, we first computed the total over the time
series. This is meant to estimate the total value scored in whatever the
index measures. For example, we assumed that the sum of the NDFI values
over the period of the 3 years considered in this study at a given pixel
gives an estimate of the total water received by that pixel. After
computing these total values scored, we then proceeded by summing-up the
total value scored by the NDII6 and the NDII7 from which we extracted
twice the total value scored by the NDFI. From the resulting value, we
then extracted twice the total value scored by the NDVI (eq. 2). The
idea is to reduce the influence of vegetation water from a total of
water contained in soil and vegetation. So doing, the remaining water
would mainly be due to the above and below-ground soil water. These
estimates were improved by adding to each the scored value of the NDFI
which is another way to estimate soil wetness. We used both the NDII6,
the NDII7 and the NDFI because to take the advantage of the potential of
the SWIR spectral domain as much as possible. This, however, is
equivalent to doubling the real overall potential of the water content.
To account for this effect, half this value was considered after
reducing the amount of water stored in vegetation which was done by
subtracting twice the NDVI to account for the influence of this
vegetation water in both the NDII6 and the NDII7 (eq. 2). Exposure to
wetness This node estimates the expected total water for each cell. It
was computed simply as the sum of the NDFI over the time series for each
pixel. In this regard, it expresses the degree to which each pixel is
relatively expose to surface water wetness. It is slightly different
from the soil water content in the sense that it does not account for
vegetation. Literally, this is mostly the surface soil water
availability / potential regardless of whether it is used or not by
vegetation. We postulate that a certain amount of water may be detected
at a given pixel at some point in time without being sustained since it
can runoff throughout porous soils shortly or stagnate nearly
permanently under saturated or impermeable soils. Contrastingly, under
soils with good water holding capacity, a relatively small amount of
water can be sustained for a relatively longer period. This implies that
the node exposure to wetness is more concerned about the hydrology than
the actual agronomy although water exposure can be a good indicator of
vegetation. It is, nonetheless, important to keep in mind that
vegetation is almost always incompatible with permanently water-logged
areas. Sensitivity to flood The node sensitivity to flood is computed as
the absolute standard error of the NDFI to estimate somewhat how fast
the flood index is changing throughout the time series. This node
describes the relationship between water and soil. Supposing that all
soil surface water originates from rainfall, and considering that this
rainfall is relatively locally constant, then the local spatial
variability in wetness can be explained by the relationship between soil
(including land cover and topography) and water. For example, from
remote sensing point of view, dry but flood-prone soils are likely to be
more sensitive to torrential rainfall event compared to permanently
saturated and water-logged soils. Vegetation sensitivity to water
variation This node was also estimated as a composite index taking into
account both the NDFI and the NDVI. Flood followed by a rapid increase
in vegetation has been used as indicator for mapping FBFS-like areas
such as flooded
rice.\textsuperscript{\protect\hyperlink{ref-Boschetti_et_al_2014}{10}}
The node describes the relationship between water and vegetation using
the ratio between these 2 metrics. For a given pixel, the idea conveyed
by this node is simply the proportion of vegetation scored relative to
amount of water received. Thus, it also implicitly gives an idea of the
soil conditions because a pair of low value NDVI (absence of vegetation)
and high value NDFI (water presence) may imply the presence of encrusted
or swampy soils where vegetation may be sensitive to water variation.
Power of tools The power of tools was estimated from the generic product
quality assessment MODLAND-wide QA bands for each pixel and date in the
time series. The MODLAND-wide QA provides a general and consistent
assessment of the quality (usability and usefulness) of the MODIS
products to inform the user on the quality, hence, the extent to which
the results of any analysis is to be appreciated. Many artefacts
(e.g.~aerosol loading, cloud, data processing algorithms, sensors
failure, view angle, etc.) can introduce errors or uncertainties in
remote sensing data, and it is therefore advised to consider such
aspects prior to their use for subsequent analysis. While the power of
tools can be estimated using band specific quality assessment, we choose
to use the MODLAND-wide QA which includes most of the artefacts in the
assessment. Table 1. Translation of 2-bit pixel level-QA in
MODLAND.\textsuperscript{\protect\hyperlink{ref-Roy_et_al_2002}{58}}
2-bit encoded per pixel QA code Decimal value Quality attribute meanings
00 0 Pixel produced, good quality, not necessary to examine more
detailed QA 01 1 Pixel produced, unreliable or unquantifiable quality,
recommend examination of more detailed QA 10 2 Pixel not produced due to
cloud effects 11 3 Pixel not produced primarily due to reasons other
than cloud In the MODLAND-wide QA framework, four quality states are
possible (Table 1) and the power of tools was estimated as the
probability of being wrong accordingly. For each pixel throughout the
time series, we remained sceptical to values greater than zero in Table
1 and the probability of being wrong was computed by dividing the count
of these values by the total length of the time series. The resulting
raster layer was then discretised to provide the node to be used in the
BNs following the procedure described in section I.5.1.2. It is worth
noting that most of the poor qualities are concentrated around water
bodies, hence the node power of tools was used for dual purpose in the
causal model despite the little spatial variability. Slope and Flow
accumulation Many approaches and algorithms for deriving hydrologic
information from topography have been proposed in diverse conceptual
forms with each having its strengths and
limitations;\textsuperscript{\protect\hyperlink{ref-Arge_et_al_2003}{24},\protect\hyperlink{ref-Jenson_and_Domingue_1988}{27},\protect\hyperlink{ref-Tarboton_1997}{59}}
mostly relying on what is known as the 3 steps conditioning
procedures.\textsuperscript{\protect\hyperlink{ref-Arge_et_al_2003}{24},\protect\hyperlink{ref-Jenson_and_Domingue_1988}{27}}
DEM conditionings are commonly used as prerequisites for computing
various topographic structures such as upslope areas, specific catchment
areas, girded networks or flow paths and are extensively discussed in
the
literature.\textsuperscript{\protect\hyperlink{ref-Arge_et_al_2003}{24},\protect\hyperlink{ref-Jenson_and_Domingue_1988}{27},\protect\hyperlink{ref-Tarboton_1997}{59},\protect\hyperlink{ref-OCallaghan_Mark_1984}{60}}
These are the filling of depressions in the original DEM, the derivation
of flow directions and the computation of flow accumulation. In this
paper, these were derived using TauDEM software version
5.3\textsuperscript{\protect\hyperlink{ref-Yang_et_al_2006}{34},\protect\hyperlink{ref-Tarboton_1997}{59},\protect\hyperlink{ref-Tarboton_et_al_1991}{61},\protect\hyperlink{ref-Tesfa_et_al_2011}{62}}
from within R. Routines were sent via system calls from R to TauDEM and
throughputs are routed from TauDEM to R to be further processed to
discrete raster data following the single layer procedure described in
section I.5.1.2. TauDEM routines along with the conceptual and
algorithmic details on the DEM conditioning are beyond the scope of this
paper. The reader is referred to TauDEM website and related
literature.\textsuperscript{\protect\hyperlink{ref-Arge_et_al_2003}{24},\protect\hyperlink{ref-Jenson_and_Domingue_1988}{27},\protect\hyperlink{ref-Tarboton_1997}{59},\protect\hyperlink{ref-Tarboton_et_al_1991}{61}--\protect\hyperlink{ref-Wallis_et_al_2009}{63}}
The original void-filled SRTM DEM (section I.2) was used in the 3 steps
normative conditioning
procedure\textsuperscript{\protect\hyperlink{ref-Jenson_and_Domingue_1988}{27}}
to compute the slope and flow accumulation layers to account for the
corresponding spatial data nodes in the BNs as described in section
I.3.1. Our approach fully accommodates for simple and looping
depressions and flat areas as suggested by Jenson \&
Domingue\textsuperscript{\protect\hyperlink{ref-Jenson_and_Domingue_1988}{27}}
and
Tarboton.\textsuperscript{\protect\hyperlink{ref-Tarboton_et_al_1991}{61}}
We computed the slope as the drop distance (tangent of the slope angle;
see TauDEM website) from which we derived the D-infinity flow direction
as the direction outwards water flows from that pixel based on the
Single-flow-direction
concept.\textsuperscript{\protect\hyperlink{ref-Arge_et_al_2003}{24},\protect\hyperlink{ref-Tarboton_1997}{59}}
Based on the derived flow directions, the flow accumulation at each
pixel was then calculated as the number of pixels draining into it.
Water presence The water presence node estimates the likelihood of
encountering water on the ground given the 3 years times series
considered in this study. Flood is perhaps the most important
characteristic of FBFS, and therefore water signature may be captured by
the sensor in these areas. We therefore used the NDFI to assess such
cases across all pixels in the study area. This was done using the
procedure described in section I.5. Temporal variation in vegetation The
temporal variation in vegetation was estimated as pixel level vegetation
anomalies using NDVI based on an approach of vegetation phenology
(Figure 2). Our approach consisted in estimating the length of the
growing season from which was then extracted an average interpolated
surface derived from it. This is based on the assumption that the
vegetation period is relatively extended under FBFS-like settings where
plants take advantages of residual moisture from previous
flooding.\textsuperscript{\protect\hyperlink{ref-VanSteenbergen_et_al_2010}{2}}
In this case, these FBFS may be found in areas having unusually higher
growing period relative to a general spatial trend.

Figure 2. Vegetation seasonality. EGS = End of growing season; BGS =
Beginning of growing season; the indices (1 and 2) indicate the first
and second seasons respectively. The NDVI has been used in many ways to
study vegetation
phenology.\textsuperscript{\protect\hyperlink{ref-DeLeeuw_et_al_2012}{64}--\protect\hyperlink{ref-Yu_et_al_2012}{66}}
We used the NDVI time series to first estimate the ratio (Figure 2) at
which the vegetation density is changing throughout the
year.\textsuperscript{\protect\hyperlink{ref-Yu_et_al_2010}{65}--\protect\hyperlink{ref-White_et_al_1997}{67}}
The NDVI ratio was calculated using the eq.
3.\textsuperscript{\protect\hyperlink{ref-Yu_et_al_2010}{65},\protect\hyperlink{ref-White_et_al_1997}{67}}
〖NDVI〗\_ratio (NDVI- 〖NDVI〗\_min )/(〖NDVI〗\_max - 〖NDVI〗\_min )
(eq.3) Where NDVIratio is the NDVI ratio of the pixel throughout the
time series ranging from 0 to 1, NDVI is the NDVI of that pixel at a
given time step in the times series, NDVImax and NDVImin are
respectively the maximal and the minimal NDVI values scored by the pixel
over the time series. The NDVI ratio (eq. 3) is consistent since its
value is normalised relative to itself. This ratio was derived
considering the 3 years period considered in the study whereby the
minimum and the maximal values were derived according to the
phenological phases (Figure 2). An NDVI ratio at a given development
stage informs on the proportion of vegetation relative to the total
attainable biomass regardless of the land cover
type.\textsuperscript{\protect\hyperlink{ref-White_et_al_1997}{67}}
While this method can help to avoid the use of locally adopted
thresholds\textsuperscript{\protect\hyperlink{ref-White_et_al_1997}{67}}
since a generic threshold above which the onset and cession of the rainy
season can be consistently estimated, deciding which ratio value to
consider as that common threshold is another question. To address this
concern, we extracted the seasonal component of the time series from
which individual seasons were separated. Based on the assumption that
the NDVI changes rapidly in the near onset and secession of the rainy
season, we then scanned each season forwards and backwards questing for
possible jumps (i.e.~± 3 standard deviation) in the values of the NDVI
ratio. When scanning forwards, then a jump corresponds to the onset
whereas the cessation is estimated scanning backwards. After locating
the dates corresponding to these onset and cessation for each cell,
their difference was calculated to estimate the length of the growing
season for each season. These raster layers were aggregated to 5 km2
interpolation surface using the \texttt{aggregate} function from the
raster package\textsuperscript{\protect\hyperlink{ref-Hijmans_2019}{46}}
to fit a thin plate spline regression model using the \texttt{Tps}
function from the fields
package.\textsuperscript{\protect\hyperlink{ref-Nychka_et_al_2018}{68}}
The model was then used to interpolate each season using the
\texttt{interpolate} from the raster package relative to the average
interpolation
surface.\textsuperscript{\protect\hyperlink{ref-Hijmans_2019}{46},\protect\hyperlink{ref-Nychka_et_al_2018}{68}}
The anomalies were estimated as the difference between the length of the
growing season and its interpolation. Finally, these anomalies were
sum-up to provide a single estimate which was then discretised to
provides nodes states for the Bayesian network as described in section
I.5.1.2. Flood at the beginning of the rainy season The beginning of the
rainy season node was defined widely from the starting of the season to
nearly the vegetation peak to account for early and late flooding. The
node flood at the beginning of the rainy season, then, estimates the
probability of getting flood during that critical period. Based on this
assumption, we extracted, from the NDFI times series, all layers
corresponding to the first 2 months from which the node states were
derived using the multilayers procedure described in section I.5.  
Results and discussion Comparative profiles of the NDSIs across
different surface features Comparative behavior of the NDSIs across
biomes. The temporal profile of different NDSIs across the main
ecological systems is provided in Figure 3. The NDSIs seem to behave
similarly across the 3 main agricultural systems (i.e.~Rainfed fields,
Sugarcane fields, FBFS fields) with more pronounced flood peaks under
FBFS fields. These peaks are also observed under flood-prone forests
where floods seem to be seasonal. Rainfed fields seems to have more
communalities (i.e.~NDSIs' scores and temporal profile) with sugarcanes
fields as do FBFS fields and flood prone forest although it is hard to
distinguish between these 4 biomes.

Figure 3. Comparative analysis and behavior of the NDSIs across
different surfaces. While one may expect NDFI values above 0.5 at least
under FBFS fields, the 3 water-related NDSIs (including the Gao NDWI,
Mcfeeters NDWI), NDFI scored negative values across the 3 different
agricultural systems making thresholds specification difficult. The
profile of vegetation (NDVI, NDII6, NDII7) under rainfed and FBFS fields
seems to present a more pronounced seasonality contrary to sugarcane
fields owing to their longer growing season resulting in moisture
stability (Gao NDWI). This vegetation seasonality along with the
extended length of the growing season becomes clearer under flood prone
forest. Interestingly, flood prone forest seems to be different from
FBFS fields with regards to vegetation density and open water stability.
While both FBFS fields and flood prone forest scored positive NDVI
values reaching up to 0.75, the NDVI values under flood prone forest
rarely drop below 0.5 contrary to FBFS where these values can drop to
0.25. Under lakes systems (shallow and deep lakes), the signatures of
the different NDSIs is rather confusing except in the case of NDVI which
seems to be unique under shallow lake. While vegetation is quasi absent
in littoral zone, there seems to be a permanent mixture of vegetation
and water in the deep water suggesting possible Eutrophication of Lake
Victoria. Forest biomes are possibly distinguishable looking at the
relative scores of the different NDSIs and their temporal profiles.
Flood-prone and moderately dense forests are more seasonal in vegetation
than dense forests exhibiting constant moisture over time. In general,
sudden drop in vegetation seems to be associated with sudden increase in
water signature under both dense and moderately dense forests while no
such association is observed under flood-prone forest and towns
suggesting the presence of open waters that are often masked by tree
canopies. While vegetation (NDVI, NDII6, NDII7) decrease from dense
forest to town through moderately dense forest as expected, the opposite
trend is observed with open water. Nonetheless, the soil moisture (Gao
NDWI) across the 3 biomes seems to be constant. Overall, most of the
NDSIs behave similarly and open water stability or the length of the
growing season can be misleading in delineating FBFS fields in the area.
Soil moisture (Gao NDWI) seems to be constant over time with little
difference across biomes suggesting the presence of good soils for FBFS
in the areas. Comparative distribution of the NDSIs values across
different surfaces. Looking at the distribution of the different NDSIs
across the different biomes (Figure 4), it is clearly difficult to
derive sharp threshold values for delineating most of the biomes,
particularly the FBFS fields. The response of open water with regards to
the different NDSIs seems to be rather odd, possibly due to the quality
of the data since poor-quality data were associated with water bodies.
Based the NDFI, the NDVI and the Mcfeeters' MDWI, FBFS fields can only
be differentiated from shallow lake but the amount of noise is yet to be
substantial considering the ranges overlapping of outliers. Considering
the NDII6, the NDII7 and the Gao's NDWI, little can be done to
distinguish FBFS fields from the other biomes due to the general lack
statistical difference.

Figure 4. Comparative distribution of the NDSIs values across different
surfaces. The best statistical difference based on the NDFI seems to be
between the signature of the shallow lake and the rainfed fields.
However, the majority of these rainfed fields seems to experience
flooding owing to their overlaps with both FBFS fields and deep lack.
This is also observed with the Mcfeeters' MDWI. In a nutshell, there is
redundancy in the scores of the NDSIs across the main biomes encountered
in the study areas, hence they might be misleading for specifying
thresholds. Consequently, there is a general lack of statistical
evidence for deriving threshold values for delineating FBFS fields from
the other type of ecological systems. Phenological overview of the
landscape texture The average trend (represented by the interquartile
range or the green medians line) depicted the phenological nature of the
vegetation in the study areas with 2 annual phenological phases. The
first phase seems to start from the beginning of April and ends during
the period end of August -- beginning of October whereas the second
starts during the period beginning of September to the middle of October
and ends in the middle of March. This main trend seems to isolate the
croplands in relation with rainfall seasonality in the area (Figure 5).
The Outliers seems also to describe further phenological differences in
vegetation across different type of biomes in the areas. Further
examination of these outliers (not shown here) as separate datasets
resulted in further temporal patterns in the NDVI values which can also
be seen in Figure 5 (black and red lines). The first pattern (upmost
outlier in black) seems to describe relatively high vegetation spots
with NDVI values ranging from 0.5 to 1 during the period February --
May. These could be due to data irregularities (sensor oversaturation
during data recording, poorly corrected data, etc.) or more intuitively
a sudden increase in vegetation due to the short rains in areas with
relatively dense vegetation such as forests. In areas with permanent
vegetation, changes in the greenness at the onset of the rainy season
can be more rapid compared to agricultural fields where a certain time
is required for crops to emerge and regreen the landscape. This is
plausible considering that these outliers disappear when considered
within the context of the upmost trend (upper whisker and upmost black
line). The second pattern (lower outliers, and black line), accounting
for vegetation indices in-between negative 0.25 and 0.5 seems to be
positively correlated with the main pattern contrary to the third
pattern (lower outliers and red line) describing vegetation indices
below negative 0.25. while it is hard to tell the land use type
described by the second pattern of these outliers, the third one is
likely to describe water and water bodies. Overall, the study area is
heterogenous and composed of several distinctive biomes with specific
phenology and vegetation density, with some of these correlated over
time.

Figure 5: Boxplot of the temporal variability in water and vegetation in
Kisumu county, Kenya. Spatio-temporal variability in water and
vegetation The analysis of the NDSIs across the main existing land use /
land cover provide little guidance to specify the boundaries of FBFS
fields. Despite the differences in water and vegetation density (Figure
5), it was interesting to look at the main type of landscape existing in
the areas and the extent to which their ecology can be appreciated in
the context of seasonality. The PCA suggests that the first 3 components
best describe the essential of the variability in vegetation supporting
the possibility of data redundancy stated earlier (section II.1.2)
whereas up to 13 were required for flood supporting the unpredictable
nature of floods (Figure 6).

Figure 6. Scree plot of the first meaningful principal components along
with the KMO and Bartlett's Test of sphericity The spatio-temporal
patterns of water and vegetation based on the 8 days MODIS composite in
the study area presented in Figure 7 and Figure 9 respectively. The
temporal variability describes the cross-pixels dominant variation over
time attributed to the PCs depicted by the PCA. These are also presented
as monthly aggregates to provide lumped view of the area. The Spatial
patterns resulting from the scores of these PCs is also provided as
maps. Based on these results,

Figure 7. Dominants spatial and temporal characteristics of flood.

Figure 8. Probability of water by ranges.

Figure 9. Dominants spatial and temporal characteristics of vegetation.

Figure 10. Probability of vegetation by ranges

Prior spatial distribution of the spatial data node states

Posterior spatial distribution of the spatial data node states

  References

\hypertarget{refs}{}
\leavevmode\hypertarget{ref-Puertas_et_al_2011}{}%
1. Puertas, D. G.-L., Steenbergen, F. van, Haile, A. M., Kool, M. \&
Embaye, T.-a. G. \emph{Flood based farming systems in Africa}. 52
(2011).

\leavevmode\hypertarget{ref-VanSteenbergen_et_al_2010}{}%
2. Steenbergen, F. van, Lawrence, P., Mehari, A., Salman, M. \& Faurès,
J.-M. \emph{Guidelines on spate irrigation}. 249 (FAO, 2010).

\leavevmode\hypertarget{ref-Varisco_1983}{}%
3. Varisco, D. M. Sayl and ghayl: The ecology of water allocation in
Yemen. \emph{Human Ecology} \textbf{11}, 365--383 (1983).

\leavevmode\hypertarget{ref-Xing_et_al_2014}{}%
4. Xing, L. \emph{et al.} China wetland extraction and classification
using MODIS data. in \emph{2014 third international workshop on earth
observation and remote sensing applications (eorsa)} 349--352 (IEEE,
2014).

\leavevmode\hypertarget{ref-FBLN_2018}{}%
5. FBLN. Flood-based Livelihood Network (FBLN) Foundation. (2018).

\leavevmode\hypertarget{ref-Haile_2010}{}%
6. Haile, A. M. \emph{A Tradition in Transition, Water Management
Reforms and Indigenous Spate Irrigation Systems in Eritrea}. 212 (CRC
Press, 2010).

\leavevmode\hypertarget{ref-Ghebreamlak_et_al_2018}{}%
7. Ghebreamlak, A. Z., Tanakamaru, H., Tada, A., Adam, B. M. \& Elamin,
K. A. E. Satellite-based mapping of cultivated area in Gash Delta Spate
Irrigation System, Sudan. \emph{Remote Sensing} \textbf{10}, 186 (2018).

\leavevmode\hypertarget{ref-Khalid_et_al_2016}{}%
8. Khalid, A. E. \emph{et al.} Performance Evaluation of Spate
Irrigation Using Remote Sensing and DEM. in \emph{The 7th international
conference on water resources and environment research (icwrer2016)}
(ICWRER2016, 2016).

\leavevmode\hypertarget{ref-Theilen-Willige_et_al_2015}{}%
9. Theilen-Willige, B. \emph{et al.} Flash Floods in the Guelmim
Area/Southwest Morocco--Use of Remote Sensing and GIS-Tools for the
Detection of Flooding-Prone Areas. \textbf{5}, 203--221 (2015).

\leavevmode\hypertarget{ref-Boschetti_et_al_2014}{}%
10. Boschetti, M., Nutini, F., Manfron, G., Brivio, P. A. \& Nelson, A.
Comparative analysis of normalised difference spectral indices derived
from MODIS for detecting surface water in flooded rice cropping systems.
\emph{PLoS ONE} \textbf{9}, (2014).

\leavevmode\hypertarget{ref-Wegmann_et_al_2016}{}%
11. Wegmann, M., Leutner, B. \& Dech, S. \emph{Remote Sensing and GIS
for Ecologists: Using Open Source Software (Data in the Wild)}. 324
(Pelagic Publishing, 2016).

\leavevmode\hypertarget{ref-Konecny_2014}{}%
12. Konecny, G. \emph{Geoinformation: remote sensing, photogrammetry and
geographic information systems}. 472 (CRC Press/Taylor \& Francis Group,
2014).

\leavevmode\hypertarget{ref-Bashari_et_al_2008}{}%
13. Bashari, H., Smith, C. \& Bosch, O. J. H. Developing decision
support tools for rangeland management by combining state and transition
models and Bayesian belief networks. \emph{Agricultural Systems}
\textbf{99}, 23--34 (2008).

\leavevmode\hypertarget{ref-Hubbard_2014}{}%
14. Hubbard, D. W. \emph{How to Measure Anything: Finding the Value of
Intangibles in Business}. 432 (John Wiley \& Sons, Inc., 2014).

\leavevmode\hypertarget{ref-Kuhnert_et_al_2010}{}%
15. Kuhnert, P. M., Martin, T. G. \& Griffiths, S. P. A guide to
eliciting and using expert knowledge in Bayesian ecological models.
\emph{Ecology Letters} \textbf{13}, 900--914 (2010).

\leavevmode\hypertarget{ref-Kuhnert_et_al_2005}{}%
16. Kuhnert, P. M., Martin, T. G., Mengersen, K. \& Possingham, H. P.
Assessing the impacts of grazing levels on bird density in woodland
habitat: A Bayesian approach using expert opinion. \emph{Environmetrics}
\textbf{16}, 717--747 (2005).

\leavevmode\hypertarget{ref-Whitney_et_al_2018}{}%
17. Whitney, C., Shepherd, K. \& Luedeling, E. \emph{Decision analysis
methods guide; Agricultural policy for nutrition}. 27 (2018).
doi:\href{https://doi.org/10.5716/WP18001.PDF}{10.5716/WP18001.PDF}

\leavevmode\hypertarget{ref-Kuhnert_2011}{}%
18. Kuhnert, P. M. Four case studies in using expert opinion to inform
priors. \emph{Environmetrics} \textbf{22}, 662--674 (2011).

\leavevmode\hypertarget{ref-Luedeling_et_al_2015}{}%
19. Luedeling, E. \emph{et al.} Fresh groundwater for Wajir - ex-ante
assessment of uncertain benefits for multiple stakeholders in a water
supply project in Northern Kenya. \emph{Frontiers in Environmental
Science} \textbf{3}, (2015).

\leavevmode\hypertarget{ref-Whitney_et_al_2018a}{}%
20. Whitney, C. \emph{et al.} Probabilistic Decision Tools for
Determining Impacts of Agricultural Development Policy on Household
Nutrition. \emph{Earth's Future} \textbf{6}, 359--372 (2018).

\leavevmode\hypertarget{ref-Yet_et_al_2016}{}%
21. Yet, B. \emph{et al.} A Bayesian Network Framework for Project Cost,
Benefit and Risk Analysis with an Agricultural Development Case Study.
\emph{Expert Systems with Applications} \textbf{60}, 141--155 (2016).

\leavevmode\hypertarget{ref-Pourret_et_al_2008}{}%
22. Pourret, O., Naïm, P. \& Marcot, B. \emph{Bayesian Networks: A
Practical Guide to Applications}. 446 (John Wiley \& Sons Ltd, 2008).

\leavevmode\hypertarget{ref-VanSteenbergen_et_al_2011}{}%
23. Steenbergen, F. van, MacAnderson, I. \& Mehari, A. \emph{Spate
irrigation in the Horn of Africa: Status and potential.} 44 (2011).

\leavevmode\hypertarget{ref-Arge_et_al_2003}{}%
24. Arge, L. \emph{et al.} Efficient flow computation on massive grid
terrain datasets. \emph{GeoInformatica} \textbf{7}, 283--313 (2003).

\leavevmode\hypertarget{ref-Gao_1996}{}%
25. Gao, B. C. NDWI - A normalized difference water index for remote
sensing of vegetation liquid water from space. \emph{Remote Sensing of
Environment} \textbf{58}, 257--266 (1996).

\leavevmode\hypertarget{ref-Hunt_and_Rock_1989}{}%
26. Hunt, E. R. \& Rock, B. N. Detection of changes in leaf water
content using Near- and Middle-Infrared reflectances. \emph{Remote
Sensing of Environment} \textbf{30}, 43--54 (1989).

\leavevmode\hypertarget{ref-Jenson_and_Domingue_1988}{}%
27. Jenson, S. K. \& Domingue, J. O. Extracting topographic structure
from digital elevation data for geographic information system analysis.
\emph{Photogrammetric Engineering and Remote Sensing} \textbf{54},
1593--1600 (1988).

\leavevmode\hypertarget{ref-Ji_et_al_2009}{}%
28. Ji, L., Zhang, L. \& Wylie, B. Analysis of Dynamic Thresholds for
the Normalized Difference Water Index. \textbf{75}, 1307--1317 (2009).

\leavevmode\hypertarget{ref-McFeeters_1996}{}%
29. McFeeters, S. K. The use of the Normalized Difference Water Index
(NDWI) in the delineation of open water features. \emph{International
Journal of Remote Sensing} \textbf{17}, 1425--1432 (1996).

\leavevmode\hypertarget{ref-Rouse_et_al_1973}{}%
30. Rouse, W., Haas, R. H. \& Deering, D. W. Monitoring vegetation
systems in the Great Plains with ERTS, NASA SP-351. in \emph{Third earth
resources technology satellite-1 symposium} 309--317 (NASA, 1973).

\leavevmode\hypertarget{ref-Roy_et_al_2014}{}%
31. Roy, D. \emph{et al.} Landsat-8: Science and product vision for
terrestrial global change research. \emph{Remote Sensing of Environment}
\textbf{145}, 154--172 (2014).

\leavevmode\hypertarget{ref-Tarboton_2003}{}%
32. Tarboton, D. Terrain analysis using digital elevation models in
hydrology. in \emph{23rd esri international users conference} (2003).

\leavevmode\hypertarget{ref-Tucker_1979}{}%
33. Tucker, C. J. Red and Photographic Infrared Linear Combinations for
Monitoring Vegetation. \emph{Remote Sensing of Environment} \textbf{08},
127--150 (1979).

\leavevmode\hypertarget{ref-Yang_et_al_2006}{}%
34. Yang, X., Chapman, G. A., Young, G. A. \& Gray, J. M. Using Compound
Topographic Index to delineate soil landscape facets from Digital
Elevation Models for comprehensive coastal assessment. in \emph{MODSIM
2005 international congress on modelling and simulation. Modelling and
simulation society of australia and new zealand} (2006).

\leavevmode\hypertarget{ref-RCoreTeam_2018}{}%
35. R Core Team. \emph{R: A language and environment for statistical
computing.} (R Foundation for Statistical Computing, 2018).

\leavevmode\hypertarget{ref-KisumuCountyGovernment_2013}{}%
36. Kisumu County Government. \emph{Kisumu County First Integrated
Development Plan 2013 - 2017.} (2013).

\leavevmode\hypertarget{ref-Farr_et_al_2000}{}%
37. Farr, T. G. \& Kobrick, M. Shuttle radar topography mission produces
a wealth of data. \emph{Eos} \textbf{81}, 583--585 (2000).

\leavevmode\hypertarget{ref-Farr_et_al_2007}{}%
38. Farr, T. G. \emph{et al.} The shuttle radar topography mission.
\emph{Reviews of Geophysics} \textbf{45}, (2007).

\leavevmode\hypertarget{ref-Nikolakopoulos_et_al_2006}{}%
39. Nikolakopoulos, K. G., Kamaratakis, E. K. \& Chrysoulakis, N. SRTM
vs ASTER elevation products. Comparison for two regions in Crete,
Greece. \emph{International Journal of Remote Sensing} \textbf{27},
4819--4838 (2006).

\leavevmode\hypertarget{ref-Jarvis_et_al_2008}{}%
40. Jarvis, A., Reuter, H. I. I., Nelson, A. \& Guevara, E. Hole-filled
seamless SRTM data V4. (2008).

\leavevmode\hypertarget{ref-Falorni_et_al_2005}{}%
41. Falorni, G., Teles, V., Vivoni, E. R., Bras, R. L. \& Amaratunga, K.
S. Analysis and characterization of the vertical accuracy of digital
elevation models from the Shuttle Radar Topography Mission.
\emph{Journal of Geophysical Research: Earth Surface} \textbf{110},
(2005).

\leavevmode\hypertarget{ref-King_et_al_1995}{}%
42. King, M. D., Herring, D. D. \& Diner, D. J. The earth observing
system (EOS): A space-based program for assessing mankind's impact on
the global environment. \emph{Opt. Photon. News} \textbf{6}, 34--39
(1995).

\leavevmode\hypertarget{ref-Vermote_2015}{}%
43. Vermote, E. MOD09A1 MODIS/Terra Surface Reflectance 8-Day L3 Global
500m SIN Grid V006. (2015).

\leavevmode\hypertarget{ref-Pearl_2000}{}%
44. Pearl, J. \emph{Causality: Models, Reasoning and Inference}. 1--386
(Cambridge University Press, 2000).
doi:\href{https://doi.org/citeulike-article-id:3888442}{citeulike-article-id:3888442}

\leavevmode\hypertarget{ref-GADM_2018}{}%
45. GADM. Database of Global Administrative Areas. (2018).

\leavevmode\hypertarget{ref-Hijmans_2019}{}%
46. Hijmans, R. J. raster: Geographic Data Analysis and Modeling.
(2019).

\leavevmode\hypertarget{ref-Mattiuzzi_and_Detsch_2018}{}%
47. Mattiuzzi, M. \& Detsch, F. MODIS: Acquisition and Processing of
MODIS Products. (2018).

\leavevmode\hypertarget{ref-Busetto_and_ranghetti_2016}{}%
48. Busetto, L. \& Ranghetti, L. MODIStsp: An R package for automatic
preprocessing of MODIS Land Products time series. \emph{Computers \&
Geosciences} \textbf{97}, 40--48 (2016).

\leavevmode\hypertarget{ref-GDAL_OGRcontributors_2018}{}%
49. contributors, G. GDAL/OGR Geospatial Data Abstraction software
Library. (2018).

\leavevmode\hypertarget{ref-Bivand_et_al_2013}{}%
50. Bivand, R. S., Pebesma, E. \& Gómez-Rubio, V. \emph{Applied Spatial
Data Analysis with R: Second Edition}. 405 (Springer New York, 2013).

\leavevmode\hypertarget{ref-Pebesma_Bivand_2005}{}%
51. Pebesma, E. \& Bivand, R. S. S classes and methods for spatial data:
the sp package. (2005).

\leavevmode\hypertarget{ref-Mardia_et_al_1979}{}%
52. Mardia, K. V., Kent, J. T. \& Bibby, J. M. \emph{Multivariate
Analysis}. 521 (Academic Press., 1979).

\leavevmode\hypertarget{ref-Pearson_1901}{}%
53. Pearson, K. On lines and planes of closest fit to systems of points
in space. \emph{Philosophical Magazine} \textbf{2}, 559--57 (1901).

\leavevmode\hypertarget{ref-Leutner_et_al_2019}{}%
54. Leutner, B., Horning, N. \& Schwalb-Willmann, J. RStoolbox: Tools
for Remote Sensing Data Analysis. (2019).

\leavevmode\hypertarget{ref-Likert_1932}{}%
55. Likert, R. A technique for the measurement of attitudes.
\emph{Archives of Psychology} \textbf{22}, 5--55 (1932).

\leavevmode\hypertarget{ref-Wickham_and_Stryjewski_2012}{}%
56. Wickham, H. \& Stryjewski, L. \emph{40 Years of Boxplots}. 1--17
(2012).

\leavevmode\hypertarget{ref-Hothorn_et_al_2006}{}%
57. Hothorn, T., Hornik, K. \& Zeileis, A. Unbiased recursive
partitioning: A conditional inference framework. \emph{Journal of
Computational and Graphical Statistics} \textbf{15}, 651--674 (2006).

\leavevmode\hypertarget{ref-Roy_et_al_2002}{}%
58. Roy, D. \emph{et al.} The MODIS Land product quality assessment
approach. \emph{Remote Sensing of Environment} \textbf{83}, 62--76
(2002).

\leavevmode\hypertarget{ref-Tarboton_1997}{}%
59. Tarboton, D. A new method for the determination of flow directions
and upslope areas in grid digital elevation models. \emph{Water
Resources Research} \textbf{33}, 309--319 (1997).

\leavevmode\hypertarget{ref-OCallaghan_Mark_1984}{}%
60. O'Callaghan, J. F. \& Mark, D. M. The extraction of drainage
networks from digital elevation data. \emph{Computer Vision, Graphics,
\& Image Processing} \textbf{23}, 323--344 (1984).

\leavevmode\hypertarget{ref-Tarboton_et_al_1991}{}%
61. Tarboton, D., Bras, R. L. \& Rodriguez‐Iturbe, I. On the extraction
of channel networks from digital elevation data. \emph{Hydrological
Processes} \textbf{5}, 81--100 (1991).

\leavevmode\hypertarget{ref-Tesfa_et_al_2011}{}%
62. Tesfa, T. K. \emph{et al.} Extraction of hydrological proximity
measures from DEMs using parallel processing. \emph{Environmental
Modelling and Software} \textbf{26}, 1696--1709 (2011).

\leavevmode\hypertarget{ref-Wallis_et_al_2009}{}%
63. Wallis, C., Watson, D., Tarboton, D. \& Wallace, R. Parallel
Flow-Direction and Contributing Area Calculation for Hydrology Analysis
in Digital Elevation Models. in \emph{The 2009 international conference
on parallel and distributed processing techniques and applications}
(2009).

\leavevmode\hypertarget{ref-DeLeeuw_et_al_2012}{}%
64. De Leeuw, J. \emph{et al.} Benefits of Riverine Water Discharge into
the Lorian Swamp, Kenya. \emph{Water} \textbf{4}, 1009--1024 (2012).

\leavevmode\hypertarget{ref-Yu_et_al_2010}{}%
65. Yu, H., Luedeling, E. \& Xu, J. Winter and spring warming result in
delayed spring phenology on the Tibetan Plateau. \emph{Pnas}
\textbf{107}, 22151--22156 (2010).

\leavevmode\hypertarget{ref-Yu_et_al_2012}{}%
66. Yu, H., Xu, J., Okuto, E. \& Luedeling, E. Seasonal Response of
Grasslands to Climate Change on the Tibetan Plateau. \emph{PLoS ONE}
\textbf{7}, (2012).

\leavevmode\hypertarget{ref-White_et_al_1997}{}%
67. White, M. A., Thornton, P. E. \& Running, S. W. A continental
phenology model for monitoring vegetation responses to interannual
climatic variability. \emph{Global Biogeochemical Cycles} \textbf{11},
217--234 (1997).

\leavevmode\hypertarget{ref-Nychka_et_al_2018}{}%
68. Nychka, D., Furrer, R., Paige, J. \& Sain, S. fields: Tools for
Spatial Data. (2018).


\end{document}
